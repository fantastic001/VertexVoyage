
\documentclass[]{amsart}


\usepackage[utf8]{inputenc}
\usepackage[T1]{fontenc}
\usepackage[serbian]{babel}
\usepackage{graphicx}
\usepackage{longtable}
\usepackage{wrapfig}
\usepackage{rotating}
\usepackage[normalem]{ulem}
\usepackage{amsmath}
\usepackage{amssymb}
\usepackage{capt-of}
\usepackage{float}
\usepackage{fancyvrb}
\usepackage{fancyhdr}

% Let us first define a custom Verbatim environment, that saves us a lot of writing
  \DefineVerbatimEnvironment{CodeListing}{Verbatim}%
    {showtabs,commandchars=\\\{\}}% showspaces and showtabs only for visualizing


\pagestyle{fancy}
\addtolength{\headheight}{\baselineskip}

\floatstyle{ruled}
\newfloat{program}{p}{}
\floatname{program}{Kod}
\fancyhf{}
\setlength{\headheight}{48pt}
\fancyhead[L]{Stefan Nožinić, II godina master studija računarskih nauka \\
Departman za matematiku i informatiku, Prirodno-matematički fakultet \\
Univerzitet u Novom Sadu. \\
}


\begin{document}

\title{VertexVoyage: Distribuirani sistem za potapanje čvorova u realnim mrežama}
\author{Stefan Nožinić \\
student II godine master studija računarskih nauka \\
Departman za matematiku i informatiku \\
Prirodno-matematički fakultet \\
Univerzitet u Novom Sadu. \\
Tip rada: master rad \\
Mentor: \MakeLowercase{prof. dr} Miloš Savić
}

\begin{abstract}
    Ovaj dokument sadrži predlog istraživačkog projekta kao završnog rada na master studijama.
    Istraživanje bi imalo za cilj implementaciju distribuiranog algoritma za potapanje čvorova u realnim mrežama gde je broj čvorova moguće skalirati na ogromne skale.
\end{abstract}


\maketitle
\newpage


\section{Uvod i opis problema}
\label{sec:introduction}

Dosta istraživanja je urađeno na temu potapanja čvorova u realnim mrežama 
sa ciljem dalje upotrebe u algoritmima mašinskog učenja i klasterisanja. \cite{grover2016node2vec}.

Realne mreže su grafovi koji imaju specifična svojstva takva da oblikuju ponašanje mreža koje se mogu naći u prurodi. Ta svojstva se ogledaju pre svega u tome da takvi grafovi imaju jednu gigantsku komponentu, postojanje habova (čvorova sa visokim stepenom) i postojanjem fenomena malog sveta odnosno svojstva da je prosečna distanca između bilo koja dva čvora mala (često manja od 10). Takođe, izuzetno bitno svojstvo je i postojanje zajednica - delova grafa gde su čvorovi jako povezani međusobno, a slabo povezani van zajednice. 

Jedan od najkorišćenijih algoritama za potapanje čvorova u realnim mrežama jeste node2vec \cite{grover2016node2vec}. Postoji dosta verzija paralelne implementacije ovog algoritma kao npr \cite{lombardo2019actornode2vec}, \cite{fang2023distributed}. Node2vec algoritam je baziran na principu po kom funkcioniše word2vec \cite{church2017word2vec}. Algoritam generiše sekvence čvorova pokretanjem nasumične šetnje po grafu, a zatim u drugoj fazi uz pomoć izgenerisanih sekvenci obučava neuronsku mrežu čija je arhitektura jedan skriveni sloj i softmax sloj na izlaznom delu. Cilj obučavanja mreže je maksimizacija verovatnoće predviđanja susednog čvora tokom nasumične šetnje za dati ulazni čvor. 

Kako bi se ceo proces paralelizovao potrebno je uraditi paralelizaciju obe faze. Za paralelizaciju nasumične šetnje potrebno je uraditi particionisanje grafa. Neki od pokušaja particionisanja su opisani u \cite{lombardo2019actornode2vec}. Paralelizacija obučavanja neuronske mreže je rađena u \cite{gupta2018distributed} i \cite{anil2018large}. 

U ovom istraživanju biće ispitani različiti modeli particionisanja realnih mreža kao i različiti pristupi obučavanju neuronske mreže. 

\section{Metode}
\label{sec:methods}

Metodologija koja će biti sprovedena je eksperimentalno istraživanje. Varijable koje će biti kontrolisane su:

\begin{itemize}
  \item Metod treniranja neuronske mreže 
  \item Metod particionisanja realnih mreža
\end{itemize}

Za potrebe istraživanja, biće izmereni sledeći parametri:

\begin{itemize}
  \item Skaliranje - odnos vremena potrebnog za potapanje na jednom i na p procesora
  \item Količina utrošenog vremena na komunikaciju između procesora 
\end{itemize}

Kako bi se verifikovala implementacija, izračunaće se korelacija između klastera dobijenih sa K-means klasterisanjem na potapanju dobijenom upotrebom standardnog node2vec algoritma i potapanju dobijenom paralelne implementacije. Ovde je očekivanje da postoji pozitivna korelacije između klastera. 



\subsection{Ulazni podaci}
\label{sec:input_data}

Algoritam će biti testiran na Zaharijevoj mreži \cite{zachary1977information} i na većim mrežama kao što je \cite{leskovec2009community}. 

Pored postojećih mreža, kao ulazni podaci će biti prosleđeni i veštački izgenerisane mreže uz pomoć stohastičkog blokovskog modela sa različitim parametrima \cite{abbe2018community}.


\bibliographystyle{unsrt}
\bibliography{./refs.bib}


\end{document}
\endinput