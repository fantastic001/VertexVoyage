
\documentclass[]{amsart}


\usepackage[utf8]{inputenc}
\usepackage[T1]{fontenc}
\usepackage[serbian]{babel}
\usepackage{graphicx}
\usepackage{longtable}
\usepackage{wrapfig}
\usepackage{rotating}
\usepackage[normalem]{ulem}
\usepackage{amsmath}
\usepackage{amssymb}
\usepackage{capt-of}
\usepackage{float}
\usepackage{fancyvrb}
\usepackage{fancyhdr}

% Let us first define a custom Verbatim environment, that saves us a lot of writing
  \DefineVerbatimEnvironment{CodeListing}{Verbatim}%
    {showtabs,commandchars=\\\{\}}% showspaces and showtabs only for visualizing


\pagestyle{fancy}
\addtolength{\headheight}{\baselineskip}

\floatstyle{ruled}
\newfloat{program}{p}{}
\floatname{program}{Kod}
\fancyhf{}
\setlength{\headheight}{48pt}
\fancyhead[L]{Stefan Nožinić, II godina master studija računarskih nauka \\
Departman za matematiku i informatiku, Prirodno-matematički fakultet \\
Univerzitet u Novom Sadu. \\
}


\begin{document}

\title{VertexVoyage: Distribuirani sistem za potapanje ;vorova u realnim mrežama}
\author{Stefan Nožinić \\
student II godine master studija računarskih nauka \\
Departman za matematiku i informatiku \\
Prirodno-matematički fakultet \\
Univerzitet u Novom Sadu. \\
Tip rada: master rad \\
Mentor: \MakeLowercase{prof. dr} Miloš Savić
}

\begin{abstract}
    Ovaj dokument sadrži predlog istraživačkog projekta kao završnog rada na master studijama.
    Istraživanje bi imalo za cilj implementaciju distribuiranog algoritma za potapanje čvorova u realnim mrežama gde je broj čvorova moguće skalirati na ogromne skale.     
\end{abstract}


\maketitle
\newpage


\section{Uvod}
\label{sec:introduction}

\cite{grover2016node2vec}
\cite{lombardo2019actornode2vec}

\subsection{Prethodna istraživanja}
\label{sec:prev_work}

\section{Metode}
\label{sec:methods}
\subsection{Ulazni podaci}
\label{sec:input_data}


\subsection{Izlazni podaci}
\label{sec:output_data}



\subsection{Implementacija}
\label{sec:implementation}

\section{Prikupljanje rezultata}
\label{sec:results}




\section{Zaključak}
\label{sec:conclusion}

\bibliographystyle{unsrt}
\bibliography{./refs.bib}


\end{document}
\endinput